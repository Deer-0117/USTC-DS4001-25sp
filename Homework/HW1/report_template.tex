\documentclass{article}
\usepackage{graphicx} % Required for inserting images
\usepackage{amsthm,amsmath,amssymb}
\usepackage[UTF8]{ctex}
\usepackage[tc]{titlepic}
\usepackage{titlesec}
\usepackage{cite}
\usepackage{fancyhdr}
\usepackage{booktabs}
\usepackage{subfigure}
\usepackage{float}
\usepackage[textwidth=14.5cm]{geometry}
\usepackage[section]{placeins}
\usepackage{makeidx}
\usepackage{mathrsfs}
\usepackage{color}
\usepackage{ulem}
\usepackage{enumitem}
\geometry{a4paper,scale=0.8,left=2cm,right=2cm}
\pagestyle{fancy}
\usepackage{microtype}
\usepackage{hyperref}
\usepackage{tcolorbox}
\usepackage{listings}

\lstset{
    language=Python, % 指定编程语言,如 Python、Java、C++ 等
    basicstyle=\footnotesize\ttfamily, % 调小字体大小
    keywordstyle=\color{blue}, % 关键字颜色
    commentstyle=\color{green!50!black}, % 注释颜色
    stringstyle=\color{red}, % 字符串颜色
    numbers=left, % 行号显示位置,left 表示在左侧显示
    numberstyle=\tiny\color{gray}, % 行号字体样式
    stepnumber=1, % 行号递增步长
    numbersep=5pt, % 行号与代码之间的间距
    backgroundcolor=\color{gray!10}, % 代码块背景颜色
    showspaces=false, % 不显示空格
    showstringspaces=false, % 不显示字符串中的空格
    showtabs=false, % 不显示制表符
    tabsize=2 % 制表符宽度
}


% 定义一个自定义命令,参数 #1 为中间的文字
\newcommand{\sectiondivider}[1]{%
  \vspace{1em} % 上下间距,可根据需要调整
  \begin{center}
    \noindent
    \makebox[0.3\linewidth]{\hrulefill}%
    \hspace{0.5em}% 左右间距
    {\Large\textbf{#1}}%
    \hspace{0.5em}%
    \makebox[0.3\linewidth]{\hrulefill}%
  \end{center}
  \vspace{3em}
}


\lhead{第 1 次作业\\\today}
\chead{中国科学技术大学\\	DS4001 人工智能原理与技术}

\rhead{Homework 1\\ {\CTEXoptions[today=old]\today}}
\newcommand{\upcite}[1]{\textsuperscript{\cite{#1}}}

\titleformat*{\section}{\bfseries\Large}
\titleformat*{\subsection}{\bfseries\large}

\title{\bfseries DS4001-25SP-HW1:搜索}
\author{姓名\quad PB22222222}

\begin{document}
%\begin{sloopypar}
\maketitle
% \clearpage

\setcounter{section}{-1}
\section{代码理解[20\%]}

\begin{enumerate}[label=(\alph*)]
    \item \textbf{[截图]} %0a
    你的截图
\end{enumerate}


\subsection{整体结构分析[6\%]}

\begin{enumerate}[label=(\alph*), start=2]
    \item \textbf{[配对]} %0b
    \begin{itemize}
        \item \texttt{(1):}你的答案
        \item \texttt{(2):}你的答案
        \item \texttt{(3):}你的答案
        \item \texttt{(4):}你的答案
        \item \texttt{(5):}你的答案
        \item \texttt{(6):}你的答案
        \item \texttt{(7):}你的答案
    \end{itemize}
\end{enumerate}

\subsection{详细代码阅读[10\%=2\%*5]}

\begin{enumerate}[label=(\alph*), start=3]
    \item \textbf{[多选]} 你的答案 %0c
    \vspace{10pt}

    \item \textbf{[单选]} 你的答案 %0d
    \vspace{10pt}

    \item \textbf{[多选]} 你的答案 %0e
    \vspace{10pt}
\end{enumerate}

\begin{enumerate}[label=(\alph*), start=6]
    \item \textbf{[多选]} 你的答案 %0f
    \vspace{10pt}

    \item  \textbf{[简答]} %0g
    你的答案 
\end{enumerate}

\section{问题 1:查找最短路径[29\%]}

\subsection{建模[10\%=6\%+4\%]}

\begin{enumerate}[label=(\alph*), start=1]
    \item \textbf{[代码]} %1a
    \begin{lstlisting}[language=Python]
    class ShortestPathProblem(SearchProblem):
    """
    Defines a search problem that corresponds to finding the shortest path 
    from `startLocation` to any location with the specified `endTag`.
    """

    def __init__(self, startLocation: str, endTag: str, cityMap: CityMap):
        self.startLocation = startLocation
        self.endTag = endTag
        self.cityMap = cityMap

    def startState(self) -> State:
        # BEGIN_YOUR_CODE 
        raise NotImplementedError("Override me")
        # END_YOUR_CODE

    def isEnd(self, state: State) -> bool:
        # BEGIN_YOUR_CODE 
        raise NotImplementedError("Override me")
        # END_YOUR_CODE

    def successorsAndCosts(self, state: State) -> List[Tuple[str, State, float]]:
        # BEGIN_YOUR_CODE 
        raise NotImplementedError("Override me")
        # END_YOUR_CODE
    \end{lstlisting}

    \item \textbf{[代码]} %1b
    \begin{lstlisting}[language=Python]
    # Your Code
    \end{lstlisting}
\end{enumerate}

\subsection{算法[19\%=6\%+5\%+2\%+6\%]}

\begin{enumerate}[label=(\alph*), start=3]
    \item \textbf{[简答]} %1c
    你的答案
    \item \textbf{[简答]} %1d
    你的答案
\end{enumerate}

\begin{enumerate}[label=(\alph*), start=5]
    \item \textbf{[判断]} %1e
    你的答案
    \item \textbf{[简答]} %1f
    你的答案
\end{enumerate}

\section{问题 2:查找带无序途径点的最短路径[17\%]}

\subsection{建模[10\%=6\%+4\%]}

\begin{enumerate}[label=(\alph*), start=1]

    \item \textbf{[代码]} %2a
    \begin{lstlisting}[language=Python]
    # Your Code
    \end{lstlisting}
    \item \textbf{[代码]} %2b
    \begin{lstlisting}[language=Python]
    # Your Code
    \end{lstlisting}
\end{enumerate}

\subsection{算法[7\%=2\%+5\%]}

\begin{enumerate}[label=(\alph*), start=3]

    \item \textbf{[判断]} %2c
    你的答案
    \item \textbf{[简答]} %2d
    你的答案
    
\end{enumerate}

\section{问题 3:使用 A* 加快搜索速度[32\%]}

\subsection{将UCS转化为A*[4\%]}

\begin{enumerate}[label=(\alph*), start=1]
    \item \textbf{[代码]} %3a
    \begin{lstlisting}[language=Python]
    # Your Code
    \end{lstlisting}
\end{enumerate}

\subsection{实现启发式函数[18\%=3\%+6\%+3\%+6\%]}

\begin{enumerate}[label=(\alph*), start=2]
    \item \textbf{[简答]} %3b
    你的答案
    \item \textbf{[代码]} %3c
    \begin{lstlisting}[language=Python]
    # Your Code
    \end{lstlisting}
\end{enumerate}

对于问题2,我们使用不带途径点的最短路径长度作为启发式函数。

\begin{enumerate}[label=(\alph*), start=4]
    \item \textbf{[简答]} %3d
    你的答案
    \item \textbf{[代码]} %3e
    \begin{lstlisting}[language=Python]
    # Your Code
    \end{lstlisting}
\end{enumerate}

\subsection{利用合肥市地图对比运行时间[10\%=4\%+6\%]}

\begin{enumerate}[label=(\alph*), start=6]
    \item \textbf{[代码]} %3f
    \begin{lstlisting}[language=Python]
    # Your Code
    \end{lstlisting}
    \item \textbf{[简答]} %3g
    你的答案
\end{enumerate}

\section*{体验反馈[2\%]}

\begin{enumerate}[label=(\alph*), start=1]
    \item \textbf{[必做]} %所花时间
    你的答案
    \item \textbf{[选做]} %其他反馈
    你的答案
\end{enumerate}


\end{document}
