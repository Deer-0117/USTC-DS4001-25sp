\documentclass{article}
\usepackage{graphicx} % Required for inserting images
\usepackage{amsthm,amsmath,amssymb}
\usepackage[UTF8]{ctex}
\usepackage[tc]{titlepic}
\usepackage{titlesec}
\usepackage{cite}
\usepackage{fancyhdr}
\usepackage{booktabs}
\usepackage{subfigure}
\usepackage{float}
\usepackage[textwidth=14.5cm]{geometry}
\usepackage[section]{placeins}
\usepackage{makeidx}
\usepackage{mathrsfs}
\usepackage{color}
\usepackage{ulem}
\usepackage{enumitem}
\geometry{a4paper,scale=0.8,left=2cm,right=2cm}
\pagestyle{fancy}
\usepackage{microtype}
\usepackage{hyperref}
\usepackage{tcolorbox}
\usepackage{listings}

\lstset{
    language=Python, % 指定编程语言,如 Python、Java、C++ 等
    basicstyle=\footnotesize\ttfamily, % 调小字体大小
    keywordstyle=\color{blue}, % 关键字颜色
    commentstyle=\color{green!50!black}, % 注释颜色
    stringstyle=\color{red}, % 字符串颜色
    numbers=left, % 行号显示位置,left 表示在左侧显示
    numberstyle=\tiny\color{gray}, % 行号字体样式
    stepnumber=1, % 行号递增步长
    numbersep=5pt, % 行号与代码之间的间距
    backgroundcolor=\color{gray!10}, % 代码块背景颜色
    showspaces=false, % 不显示空格
    showstringspaces=false, % 不显示字符串中的空格
    showtabs=false, % 不显示制表符
    tabsize=2 % 制表符宽度
}


% 定义一个自定义命令,参数 #1 为中间的文字
\newcommand{\sectiondivider}[1]{%
  \vspace{1em} % 上下间距,可根据需要调整
  \begin{center}
    \noindent
    \makebox[0.3\linewidth]{\hrulefill}%
    \hspace{0.5em}% 左右间距
    {\Large\textbf{#1}}%
    \hspace{0.5em}%
    \makebox[0.3\linewidth]{\hrulefill}%
  \end{center}
  \vspace{3em}
}


\lhead{第 2 次作业\\\today}
\chead{中国科学技术大学\\	DS4001 人工智能原理与技术}

\rhead{Homework 2\\ {\CTEXoptions[today=old]\today}}
\newcommand{\upcite}[1]{\textsuperscript{\cite{#1}}}

\titleformat*{\section}{\bfseries\Large}
\titleformat*{\subsection}{\bfseries\large}

\title{\bfseries DS4001-25SP-HW2:搜索}
\author{姓名\quad PB}

\begin{document}
%\begin{sloopypar}
\maketitle
% \clearpage


% Problem 1
\section{问题 1:马尔可夫决策过程[9\%=6\%+3\%]}

\begin{enumerate}[label=(\alph*), start=1]
    
    \item TODO %1a
    
    \item TODO %1b

\end{enumerate}
\

% Problem 2
\section{问题 2:Q-Learning[12\%=3\%+6\%+3\%]}

\begin{enumerate}[label=(\alph*), start=1]

    \item TODO %2a
    
    \item TODO %2b
    
    \item TODO %2c
    
\end{enumerate}
\

% Problem 3
\section{问题 3:Gobang Programming[53\%=33\%+10\%+10\%]}


\begin{enumerate}[label=(\alph*), start=1]
    \item \textbf{[代码]} %3a
    % get_next_state 5分
    \begin{lstlisting}[language=Python]
def get_next_state(self, action: Tuple[int, int, int], noise: Tuple[int, int, int]):
    # BEGIN_YOUR_CODE (our solution is 3 line of code, but don't worry if you deviate from this)

    # END_YOUR_CODE

    if noise is not None:
        white, x_white, y_white = noise
        next_state[x_white][y_white] = white
    return next_state
    \end{lstlisting}
    
    % sample_noise 3分
    \begin{lstlisting}[language=Python]
def sample_noise(self):
    if self.action_space:
        # BEGIN_YOUR_CODE (our solution is 2 line of code, but don't worry if you deviate from this)

        # END_YOUR_CODE
        return 2, x, y
    else:
        return None
    \end{lstlisting}

    % get_connection_and_reward 5分
    \begin{lstlisting}[language=Python]
def get_connection_and_reward(self, action: Tuple[int, int, int], noise: Tuple[int, int, int]):
    # BEGIN_YOUR_CODE (our solution is 4 line of code, but don't worry if you deviate from this)

    # END_YOUR_CODE

    return black_1, white_1, black_2, white_2, reward
    \end{lstlisting}

    % sample_action_and_noise 8分
    \begin{lstlisting}[language=Python]
def sample_action_and_noise(self, eps: float):
    # BEGIN_YOUR_CODE (our solution is 8 line of code, but don't worry if you deviate from this)

    # END_YOUR_CODE
    return action, self.sample_noise()
    \end{lstlisting}

    % q_learning_update 12分
    \begin{lstlisting}[language=Python]
def q_learning_update(self, s0_: np.array, action: Tuple[int, int, int], s1_: np.array, reward: float,
                          alpha_0: float = 1):
    s0, s1 = self.array_to_hashable(s0_), self.array_to_hashable(s1_)
    self.s_a_visited[(s0, action)] = 1 if (s0, action) not in self.s_a_visited else \
        self.s_a_visited[(s0, action)] + 1
    alpha = alpha_0 / self.s_a_visited[(s0, action)]

    # BEGIN_YOUR_CODE (our solution is 18 line of code, but don't worry if you deviate from this)

    # END_YOUR_CODE
    \end{lstlisting}

    \item TODO % 3b

    \item TODO % 3c
\end{enumerate}
\

% Problem 4
\section{问题 4:Deeper Understanding[16\%=5\%+5\%+2\%+4\%]}
\subsection{Bellman 算子与压缩映射}
\begin{enumerate}[label=(\alph*), start=1]
    
    \item TODO % 4.1 a
    
    \item TODO % 4.1 b
    
\end{enumerate}

\subsection{重要性采样}
\begin{enumerate}[label=(\alph*), start=1]
    
    \item TODO % 4.2 a
    
    \item TODO % 4.2 b
    
\end{enumerate}
\

\section*{体验反馈[10\%]}

\begin{enumerate}[label=(\alph*), start=1]
    \item \textbf{[必做]} %所花时间
    你的答案
    \item \textbf{[选做]} %其他反馈
    你的答案
\end{enumerate}


\end{document}
