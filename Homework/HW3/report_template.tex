\documentclass{article}
\usepackage{graphicx} % Required for inserting images
\usepackage{amsthm,amsmath,amssymb}
\usepackage[UTF8]{ctex}
\usepackage[tc]{titlepic}
\usepackage{titlesec}
\usepackage{cite}
\usepackage{fancyhdr}
\usepackage{booktabs}
\usepackage{subfigure}
\usepackage{float}
\usepackage[textwidth=14.5cm]{geometry}
\usepackage[section]{placeins}
\usepackage{makeidx}
\usepackage{mathrsfs}
\usepackage{color}
\usepackage{ulem}
\usepackage{enumitem}
\geometry{a4paper,scale=0.8,left=2cm,right=2cm}
\pagestyle{fancy}
\usepackage{microtype}
\usepackage{hyperref}
\usepackage{tcolorbox}
\usepackage{listings}

\lstset{
    language=Python, % 指定编程语言,如 Python、Java、C++ 等
    basicstyle=\footnotesize\ttfamily, % 调小字体大小
    keywordstyle=\color{blue}, % 关键字颜色
    commentstyle=\color{green!50!black}, % 注释颜色
    stringstyle=\color{red}, % 字符串颜色
    numbers=left, % 行号显示位置,left 表示在左侧显示
    numberstyle=\tiny\color{gray}, % 行号字体样式
    stepnumber=1, % 行号递增步长
    numbersep=5pt, % 行号与代码之间的间距
    backgroundcolor=\color{gray!10}, % 代码块背景颜色
    showspaces=false, % 不显示空格
    showstringspaces=false, % 不显示字符串中的空格
    showtabs=false, % 不显示制表符
    tabsize=2 % 制表符宽度
}


% 定义一个自定义命令,参数 #1 为中间的文字
\newcommand{\sectiondivider}[1]{%
  \vspace{1em} % 上下间距,可根据需要调整
  \begin{center}
    \noindent
    \makebox[0.3\linewidth]{\hrulefill}%
    \hspace{0.5em}% 左右间距
    {\Large\textbf{#1}}%
    \hspace{0.5em}%
    \makebox[0.3\linewidth]{\hrulefill}%
  \end{center}
  \vspace{3em}
}


\lhead{第 3 次作业\\\today}
\chead{中国科学技术大学\\	DS4001 人工智能原理与技术}

\rhead{Homework 3\\ {\CTEXoptions[today=old]\today}}
\newcommand{\upcite}[1]{\textsuperscript{\cite{#1}}}

\titleformat*{\section}{\bfseries\Large}
\titleformat*{\subsection}{\bfseries\large}

\title{\bfseries DS4001-25SP-HW3:贝叶斯网络}
\author{姓名\quad PB00000001}

\begin{document}
%\begin{sloopypar}
\maketitle
% \clearpage

\section{马尔可夫链[24\%]}
\subsection{回答问题[20\%]}
\subsubsection{手动计算[4\%]}

\begin{enumerate}
    
    \item[1.] TODO 
    
    \item[2.] TODO 

    \item[3.] TODO 

\end{enumerate}

\subsubsection{代码填空与参数估计[16\%]}
TODO:code + figure


\subsubsection{运行结果对比[4\%]}
TODO



\section{贝叶斯网络[64\%]}
\subsection{贝叶斯网络:推理[54\%]}
\subsubsection{精确推理[16\%]}

\begin{lstlisting}[language=Python]
your code
\end{lstlisting}



\subsubsection{模糊推理[16\%]}

\begin{lstlisting}[language=Python]
your code
\end{lstlisting}




\subsubsection{粒子滤波[16\%]}

\begin{lstlisting}[language=Python]
your code
\end{lstlisting}



\subsubsection{思考题[6\%]}
TODO

\subsection{贝叶斯网络:学习[10\%]}
TODO

\section{贝叶斯深度学习[6\%]}

TODO




\section*{体验反馈[6\%]}

\begin{enumerate}[label=(\alph*), start=1]
    \item \textbf{[必做]} %所花时间
    TODO
    \item \textbf{[选做]} %其他反馈
    TODO
\end{enumerate}


\end{document}
